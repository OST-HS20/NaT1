\section{LTI-System}
LTI-Systeme sind idealisierte Modelle mit Linearität und Zeitinvarianz. Der Operator $T$ bildet $x(t)$ am Eingang auf $y(t)$ ab.

\subsection{Eigenschaften}
Ein LTI-System hat folgende Eigenschaften im \textbf{Zeitbereich}:
\begin{align*}
	T[x_1(t) + x_2(t)] &= y_1(t) + y_2(t) \\
	T[a\cdot x(t)] &= a\cdot y(t) \\
	T[x(t - t_0)] &= y(t- t_0)
\end{align*}

\noindent Um das Ausgangssignal $y(t)$ mittels der Übertragungsfunktion $h(t)$ und dem Eingangssignal $x(t)$ zu berechnen:
\begin{align*}
	y(t) = x(t) * h(t) &\transform 	Y(\omega) = X(\omega) \cdot H(\omega)
\end{align*}


